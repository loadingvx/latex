%-------------------------------------------------------------------------------
% 简历风格
%-------------------------------------------------------------------------------
\documentclass[a4paper,nolmodern]{moderncv}
\moderncvtheme[orange]{classic}

%-------------------------------------------------------------------------------
% 包含sty文件中的设定
% 1. 中文支持
% 2. 行间距设定
% 3. 字体设定
% 4. 编码设定
%-------------------------------------------------------------------------------
\usepackage{config}


%-------------------------------------------------------------------------------
% 个人信息设定
%-------------------------------------------------------------------------------
% 个人经历起止时间
\usepackage[firstyear=2005,lastyear=2015]{moderntimeline}
%\title{工程师}
\firstname{梁成明}
\familyname{}
\mobile{18612275838}
%\fax{fax (optional)}
\email{loadingvx@163.com}
\homepage{www.liangchengming.com}
\photo[80pt]{identity}

\extrainfo{%
\octocat~\httplink{www.github.com/loadingvx}\\%
}

\myquote{工作以来常用python和c语言. 在当当工作3年, 主要工作为服务器端开发,
有大数据工具使用经验,  维护并重构当当网推荐后台及其他系统,有少量机器学习工作}
{}
%\myquote{踏实做人}{liangchengming}

%-------------------------------------------------------------------------------
% 简历正文
%-------------------------------------------------------------------------------

\begin{document}
\hyphenpenalty=10000

\maketitle

\section{技能}

\cvcomputer{主要语言}{C, Python, C++, markdown}
           {常用工具}{Vim}
\cvcomputer{常用框架}{libevhtp(http), Hadoop, spark}
           {数据存储}{MySQL, HDFS, Redis}

\cvcomputer{代码管理}{GIT, SVN}
           {常用系统}{Linux CentOS}

%\devnotes{Developer}{Contributor}

\section{教育和经历}
\tlcventry{2012}{0}{北京当当网信息技术有限公司}{营销系统开发部}{个性化推荐和算法组}{开发工程师} {%
}

\tlcventry{2009}{2012}{北京航空航天大学}{NLSDE国家重点实验室}{计算机技术}{硕士学位} {%
}

\tlcventry{2005}{2009}{山东大学(威海)}{~}{软件工程}{学士学位} {%
}

\section{工作内容}

\tldatecventry{2014.10}{算法模型开发}{}{LR/Spark}{python,spark}{%
\begin{tightitemize}
 \item ~迁移算法程序到Spark
 \item ~共4人负责算法模型, 使用python语言开发, 个人负责样本生成模块.
 \item ~LR模型, 由liblinear迁移spark完毕,工程持续越1个月. AUC保持在0.71
\end{tightitemize}}

\tldatecventry{2014.8}{后台开发}{}{广告后台}{C/C++, python HTTP}{%
\begin{tightitemize}
 \item ~维护当当单品页面广告后台服务
 \item ~带领另外1人, 重构部分代码.
\end{tightitemize}}

\tldatecventry{2013.2}{后台开发}{}{推荐后台}{C/C++, golang HTTP}{%
\begin{tightitemize}
 \item 2012-2013 推荐系统后台服务的开发和维护, 使用c/c++, 基于Libevhtp架构.
 \item 2012-2013 设计实现了基于Redis的商品实时数据API服务, QPS 5000左右,
 \begin{tightitemize}
   \item 目前已经发展为个性化推荐组和广告组, 相关前台的基础API. 后交付他人维护.
 \end{tightitemize}
 \item 2013-2014 重写后台新架构(golang), 重构数据端离线作业和调度
 \begin{tightitemize}
   \item 服务支持了简单灰度发布,算法对比. 可在线实时切分流量, 增加数据回路的支持, log支持hive数据格式.
   \item 推荐服务日PV约600-800万, 响应时间2ms-5ms
   \item 期间参与部分推荐算法数据的生成(3人),简单使用Apriori计算候选集,并应用在购物车页面.
 \end{tightitemize}
\end{tightitemize}
}

\tldatecventry{2012.6}{大数据开发}{}{Hadoop后台}{python}{%
\begin{tightitemize}
 \item 之后主要负责大数据相关基础数据生成
 \item 当当用户的点击浏览数据统计,由log数据生成树形结构, 为算法提供基础行为数据.
 \item 后交付他人维护.数据每日更新, 供推荐流程使用至今.
\end{tightitemize}}

\tldatecventry{2012.2}{单点登陆}{内部系统}{}{Java/LDAP}{%
\begin{tightitemize}
 \item 入职当当网之后的第一份工作为整合内部各个子系统的入口, 做单点登陆.
 \item 设计方案,并实现了基于SSL验证的统一登陆接口(Java),供所有子系统调用.(基于LDAP协议)
 \item 后交付他人维护, 目前已经对接越20多个子系统.
\end{tightitemize}}


\section{Foreign Languages}
\cvlanguage{英语}{Good}{}


\end{document}




